\chapter{\ifenglish Conclusions and Discussions\else บทสรุปและข้อเสนอแนะ\fi}

\section{\ifenglish Conclusions\else สรุปผล\fi}

\section{\ifenglish Challenges\else ปัญหาที่พบและแนวทางการแก้ไข\fi}

ในการทำโครงงานนี้ พบว่าเกิดปัญหาหลักๆ ดังนี้
\begin{itemize}
    \item ปัญหาเกี่ยวกับปริมาณการใช้พลังงานของอุปกรณ์ที่มากเกินไป
    \item ปัญหาเกี่ยวกับ GPS ที่ระบุตำแหน่งได้ยาก เนื่องจากอุปกรณ์อยู่ในรถโดยสาร แต่เสา GPS จำเป็นจะต้องอยู่ด้านนอก แต่ทำไม่ได้
    \item การพัฒนาระบบของการแสดงผลนั้นพบว่ามีปัญหาในเรื่องของการรับส่งข้อมูลระหว่างอุปกรณ์ IoT และทาง server จึงมีความล่าช้า
    \item ทางระบบ cloud ที่ได้เปิดขึ้นนั้นมีขนาดความจุที่อาจจะไม่เพียงพอเนื่องจากได้นำระบบทั้งหมดไปส่งไว้ที่ cloud ทำให้ระบบ cloud นั้นอาจจะล่มได้
  \end{itemize}
\section{\ifenglish%
Suggestions and further improvements
\else%
ข้อเสนอแนะและแนวทางการพัฒนาต่อ
\fi
}

ข้อเสนอแนะเพื่อพัฒนาโครงงานนี้ต่อไป มีดังนี้
\begin{itemize}
    \item การลดปริมาณการใช้พลังงานของอุปกรณ์ อาทิ นอกจากปิดโมเดมระหว่างที่ไม่ใช้งานแล้ว ให้ปิดอุปกรณ์วัดระยะห่างด้วย
    \item พัฒนาให้ใช้ร่วมกับคอมพิวเตอร์บนรถโดยสาร เนื่องจากมีระบบไฟที่พร้อมกว่า รวมไปถึงได้เชื่อมต่อเครือข่ายและมี GPS ในตัวเรียบร้อยแล้ว
    \item พัฒนาให้ดูดี แข็งแรง สามารถติดตั้งได้อย่างมั่นคง และสามารถใช้งานได้ง่าย
    \item สามารถนำไปพัฒนาเพื่อรองรับได้หลายสายขนส่งพร้อมกันได้
    \item สามารถนำไปพัฒนาเว็บไซต์เพื่อเพิ่มความสวยงามและใช้งานได้ดีขึ้น
  \end{itemize}