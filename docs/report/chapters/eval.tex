\chapter{\ifproject%
\ifenglish Experimentation and Results\else การทดลองและผลลัพธ์\fi
\else%
\ifenglish System Evaluation\else การประเมินระบบ\fi
\fi}

ในบทนี้จะกล่าวถึงผลการทดลองจากการวัดผลส่วนต่างๆของโครงงาน ดังนี้

\section{ส่วนอุปกรณ์วัดความหนาแน่น}

โดยได้ติดตั้งอุปกรณ์ทดสอบบนรถไฟฟ้าของขนส่งมวลชนมหาวิทยาลัยเชียงใหม่ ในช่วงวันที่ 25 มีนาคม 2567 ถึงวันที่ 29 มีนาคม 2567 โดยติดตั้งบนรถสายที่ 3 โดยสารหมายเลข 33 มีเวลาให้บริการตั้งแต่ 09:10 น. ถึง 22:00 น.

\subsection{ความแม่นยำของข้อมูล}

เปรียบเทียบข้อมูลที่ได้จากระบบเทียบกับข้อมูลจริงของผู้โดยสาร โดยการนั่งโดยสารและนับความหนาแน่นจริงบนรถโดยสาร เทียบกับค่าที่ขึ้นในระบบ ทดสอบกับช่วงเวลา 2 ชั่วโมงโดยประมาณ เป็นเวลา 2 วัน โดยมีผลการทดลองดังนี้

\subsubsection{วันที่ 27 มีนาคม 2567}
    \textbf{เวลา 12:35 น.}
    {\tiny\begin{center}
        \begin{tabular}{ | c | c | c | c | c | c | c | c | c | c | c | c | c |  }
            \hline
                \multirow{2}{*}{สถานี} & \multicolumn{3}{|c|}{ความหนาแน่นจริง (คน)} & \multicolumn{3}{|c|}{ความหนาแน่นบนระบบ (คน)} & \multicolumn{3}{|c|}{ผลต่าง (คน)} & \multicolumn{3}{|c|}{ร้อยละที่คลาดเคลื่อน (\%)} \\
            \hline
                & เข้า & ออก & ปัจจุบัน & เข้า & ออก & ปัจจุบัน & เข้า & ออก & ปัจจุบัน & เข้า & ออก & ปัจจุบัน \\
            \hline
                สถานีกลางรถไฟฟ้า ขสมช.            & 4 & 0 & 4 & 3 & 0 & 3 & 1 & 0 & 1 & 6.25 & 0 & 6.25 \\
                อาคารปฏิบัติการกลางคณะวิทยาศาสตร์    & 6 & 0 & 6 & 4 & 0 & 4 & 2 & 0 & 2 & 12.5 & 0 & 12.5 \\
                โรงอาหารคณะรัฐศาสตร์ (ตรงข้าม)*     & 6 & 4 & 2 & 4 & 0 & 4 & 2 & 4 & 2 & 12.5 & 25 & 12.5 \\
                ไปรษณีย์                          & 6 & 4 & 2 & 4 & 0 & 4 & 2 & 4 & 2 & 12.5 & 25 & 12.5 \\ 
                ลานจอดรถ อ่างแก้ว                  & 6 & 4 & 2 & 4 & 0 & 4 & 2 & 4 & 2 & 12.5 & 25 & 12.5 \\
                โรงอาหารคณะมนุษยศาสตร์            & 6 & 4 & 2 & 4 & 0 & 4 & 2 & 4 & 2 & 12.5 & 25 & 12.5 \\
                อาคาร HB7 คณะมนุษยศาสตร์          & 6 & 6 & 0 & 4 & 1 & 3 & 2 & 5 & 3 & 12.5 & 31.25 & 18.75 \\
                สำนักหอสมุด                       & 6 & 6 & 0 & 4 & 1 & 3 & 2 & 5 & 3 & 12.5 & 31.25 & 18.75 \\
                สถานีกลางรถไฟฟ้า ขสมช.            & 6 & 6 & 0 & 4 & 1 & 3 & 2 & 5 & 3 & 12.5 & 31.25 & 18.75 \\
            \hline
        \end{tabular}
    \end{center}}
    * ตั้งค่าสถานีผิด จึงทำให้ค่าความหนาแน่นบนระบบไม่ถูกต้อง ได้แก้ไขแล้วในรอบถัดไป

    \textbf{เวลา 12:55 น.}
    {\tiny\begin{center}
        \begin{tabular}{ | c | c | c | c | c | c | c | c | c | c | c | c | c |  }
            \hline
                \multirow{2}{*}{สถานี} & \multicolumn{3}{|c|}{ความหนาแน่นจริง (คน)} & \multicolumn{3}{|c|}{ความหนาแน่นบนระบบ (คน)} & \multicolumn{3}{|c|}{ผลต่าง (คน)} & \multicolumn{3}{|c|}{ร้อยละที่คลาดเคลื่อน (\%)} \\
            \hline
                & เข้า & ออก & ปัจจุบัน & เข้า & ออก & ปัจจุบัน & เข้า & ออก & ปัจจุบัน & เข้า & ออก & ปัจจุบัน \\
            \hline
                สถานีกลางรถไฟฟ้า ขสมช.            & 0 & 0 & 0 & 0 & 0 & 0 & 0 & 0 & 0 & 0 & 0 & 0 \\
                อาคารปฏิบัติการกลางคณะวิทยาศาสตร์    & 0 & 0 & 0 & 0 & 0 & 0 & 0 & 0 & 0 & 0 & 0 & 0 \\
                โรงอาหารคณะรัฐศาสตร์ (ตรงข้าม)      & 0 & 0 & 0 & 0 & 0 & 0 & 0 & 0 & 0 & 0 & 0 & 0 \\
                ไปรษณีย์                          & 0 & 0 & 0 & 0 & 0 & 0 & 0 & 0 & 0 & 0 & 0 & 0 \\
                ลานจอดรถ อ่างแก้ว                  & 0 & 0 & 0 & 0 & 0 & 0 & 0 & 0 & 0 & 0 & 0 & 0 \\
                โรงอาหารคณะมนุษยศาสตร์            & 0 & 0 & 0 & 0 & 0 & 0 & 0 & 0 & 0 & 0 & 0 & 0 \\
                อาคาร HB7 คณะมนุษยศาสตร์          & 0 & 0 & 0 & 0 & 0 & 0 & 0 & 0 & 0 & 0 & 0 & 0 \\
                สำนักหอสมุด                       & 0 & 0 & 0 & 0 & 0 & 0 & 0 & 0 & 0 & 0 & 0 & 0 \\
                สถานีกลางรถไฟฟ้า ขสมช.            & 0 & 0 & 0 & 0 & 0 & 0 & 0 & 0 & 0 & 0 & 0 & 0 \\
            \hline
        \end{tabular}
    \end{center}}

    \textbf{เวลา 13:15 น.}
    {\tiny\begin{center}
        \begin{tabular}{ | c | c | c | c | c | c | c | c | c | c | c | c | c |  }
            \hline
                \multirow{2}{*}{สถานี} & \multicolumn{3}{|c|}{ความหนาแน่นจริง (คน)} & \multicolumn{3}{|c|}{ความหนาแน่นบนระบบ (คน)} & \multicolumn{3}{|c|}{ผลต่าง (คน)} & \multicolumn{3}{|c|}{ร้อยละที่คลาดเคลื่อน (\%)} \\
            \hline
                & เข้า & ออก & ปัจจุบัน & เข้า & ออก & ปัจจุบัน & เข้า & ออก & ปัจจุบัน & เข้า & ออก & ปัจจุบัน \\
            \hline
                สถานีกลางรถไฟฟ้า ขสมช.            & 1 & 0 & 1 & 0 & 0 & 0 & 1 & 0 & 1 & 6.25 & 0 & 6.25 \\
                อาคารปฏิบัติการกลางคณะวิทยาศาสตร์    & 1 & 0 & 1 & 0 & 0 & 0 & 1 & 0 & 1 & 6.25 & 0 & 6.25 \\
                โรงอาหารคณะรัฐศาสตร์ (ตรงข้าม)      & 1 & 0 & 1 & 0 & 0 & 0 & 1 & 0 & 1 & 6.25 & 0 & 6.25 \\
                ไปรษณีย์                          & 1 & 0 & 1 & 0 & 0 & 0 & 1 & 0 & 1 & 6.25 & 0 & 6.25 \\
                ลานจอดรถ อ่างแก้ว                  & 1 & 0 & 1 & 0 & 0 & 0 & 1 & 0 & 1 & 6.25 & 0 & 6.25 \\
                โรงอาหารคณะมนุษยศาสตร์            & 1 & 0 & 1 & 0 & 0 & 0 & 1 & 0 & 1 & 6.25 & 0 & 6.25 \\
                อาคาร HB7 คณะมนุษยศาสตร์          & 1 & 0 & 1 & 0 & 0 & 0 & 1 & 0 & 1 & 6.25 & 0 & 6.25 \\
                สำนักหอสมุด                       & 1 & 1 & 0 & 0 & 0 & 0 & 0 & 1 & 0 & 0 & 6.25 & 0 \\
                สถานีกลางรถไฟฟ้า ขสมช.            & 1 & 1 & 0 & 0 & 0 & 0 & 0 & 1 & 0 & 0 & 6.25 & 0 \\
            \hline
        \end{tabular}
    \end{center}}

    เฉลี่ยแล้วกิจกรรมการเข้า-ออกรวมคลาดเคลื่อนเฉลี่ยร้อยละ \textbf{1.89} และ จำนวนผู้โดยสารบนรถคลาดเคลื่อนเฉลี่ยร้อยละ \textbf{6.25}

\subsubsection{วันที่ 28 มีนาคม 2567}
    \textbf{เวลา 09:00 น.}
    {\tiny\begin{center}
        \begin{tabular}{ | c | c | c | c | c | c | c | c | c | c | c | c | c |  }
            \hline
                \multirow{2}{*}{สถานี} & \multicolumn{3}{|c|}{ความหนาแน่นจริง (คน)} & \multicolumn{3}{|c|}{ความหนาแน่นบนระบบ (คน)} & \multicolumn{3}{|c|}{ผลต่าง (คน)} & \multicolumn{3}{|c|}{ร้อยละที่คลาดเคลื่อน (\%)} \\
            \hline
                & เข้า & ออก & ปัจจุบัน & เข้า & ออก & ปัจจุบัน & เข้า & ออก & ปัจจุบัน & เข้า & ออก & ปัจจุบัน \\
            \hline
                สถานีกลางรถไฟฟ้า ขสมช.            & 0 & 0 & 0 & 0 & 0 & 0 & 0 & 0 & 0 & 0 & 0 & 0 \\
                อาคารปฏิบัติการกลางคณะวิทยาศาสตร์    & 0 & 0 & 0 & 0 & 0 & 0 & 0 & 0 & 0 & 0 & 0 & 0 \\
                โรงอาหารคณะรัฐศาสตร์ (ตรงข้าม)      & 0 & 0 & 0 & 0 & 0 & 0 & 0 & 0 & 0 & 0 & 0 & 0 \\
                ไปรษณีย์                          & 0 & 0 & 0 & 0 & 0 & 0 & 0 & 0 & 0 & 0 & 0 & 0 \\
                ลานจอดรถ อ่างแก้ว*                 & 0 & 0 & 0 & - & - & - & - & - & - & - & - & - \\
                โรงอาหารคณะมนุษยศาสตร์            & 0 & 0 & 0 & 0 & 0 & 0 & 0 & 0 & 0 & 0 & 0 & 0 \\
                อาคาร HB7 คณะมนุษยศาสตร์          & 0 & 0 & 0 & 0 & 0 & 0 & 0 & 0 & 0 & 0 & 0 & 0 \\
                สำนักหอสมุด                       & 2 & 0 & 2 & 2 & 0 & 2 & 0 & 0 & 0 & 0 & 0 & 0 \\
                สถานีกลางรถไฟฟ้า ขสมช.            & 2 & 2 & 0 & 2 & 2 & 0 & 0 & 0 & 0 & 0 & 0 & 0 \\
            \hline
        \end{tabular}
    \end{center}}
    * ตั้งค่าสถานีผิด จึงทำให้ไม่มีข้อมูลส่งออกมา ได้แก้ไขแล้วในรอบถัดไป

    \textbf{เวลา 09:30 น.}
    {\tiny\begin{center}
        \begin{tabular}{ | c | c | c | c | c | c | c | c | c | c | c | c | c |  }
            \hline
                \multirow{2}{*}{สถานี} & \multicolumn{3}{|c|}{ความหนาแน่นจริง (คน)} & \multicolumn{3}{|c|}{ความหนาแน่นบนระบบ (คน)} & \multicolumn{3}{|c|}{ผลต่าง (คน)} & \multicolumn{3}{|c|}{ร้อยละที่คลาดเคลื่อน (\%)} \\
            \hline
                & เข้า & ออก & ปัจจุบัน & เข้า & ออก & ปัจจุบัน & เข้า & ออก & ปัจจุบัน & เข้า & ออก & ปัจจุบัน \\
            \hline
                สถานีกลางรถไฟฟ้า ขสมช.            & 0 & 0 & 0 & 0 & 0 & 0 & 0 & 0 & 0 & 0 & 0 & 0 \\
                อาคารปฏิบัติการกลางคณะวิทยาศาสตร์    & 0 & 0 & 0 & 0 & 0 & 0 & 0 & 0 & 0 & 0 & 0 & 0 \\
                โรงอาหารคณะรัฐศาสตร์ (ตรงข้าม)      & 0 & 0 & 0 & 0 & 0 & 0 & 0 & 0 & 0 & 0 & 0 & 0 \\
                ไปรษณีย์                          & 0 & 0 & 0 & 0 & 0 & 0 & 0 & 0 & 0 & 0 & 0 & 0 \\
                ลานจอดรถ อ่างแก้ว                  & 0 & 0 & 0 & 0 & 0 & 0 & 0 & 0 & 0 & 0 & 0 & 0 \\
                โรงอาหารคณะมนุษยศาสตร์            & 0 & 0 & 0 & 0 & 0 & 0 & 0 & 0 & 0 & 0 & 0 & 0 \\
                อาคาร HB7 คณะมนุษยศาสตร์          & 0 & 0 & 0 & 0 & 0 & 0 & 0 & 0 & 0 & 0 & 0 & 0 \\
                สำนักหอสมุด                       & 0 & 0 & 0 & 0 & 0 & 0 & 0 & 0 & 0 & 0 & 0 & 0 \\
                สถานีกลางรถไฟฟ้า ขสมช.            & 0 & 0 & 0 & 0 & 0 & 0 & 0 & 0 & 0 & 0 & 0 & 0 \\
            \hline
        \end{tabular}
    \end{center}}

    \textbf{เวลา 10:00 น.}
    {\tiny\begin{center}
        \begin{tabular}{ | c | c | c | c | c | c | c | c | c | c | c | c | c |  }
            \hline
                \multirow{2}{*}{สถานี} & \multicolumn{3}{|c|}{ความหนาแน่นจริง (คน)} & \multicolumn{3}{|c|}{ความหนาแน่นบนระบบ (คน)} & \multicolumn{3}{|c|}{ผลต่าง (คน)} & \multicolumn{3}{|c|}{ร้อยละที่คลาดเคลื่อน (\%)} \\
            \hline
                & เข้า & ออก & ปัจจุบัน & เข้า & ออก & ปัจจุบัน & เข้า & ออก & ปัจจุบัน & เข้า & ออก & ปัจจุบัน \\
            \hline
                สถานีกลางรถไฟฟ้า ขสมช.            & 0 & 0 & 0 & 0 & 0 & 0 & 0 & 0 & 0 & 0 & 0 & 0 \\
                อาคารปฏิบัติการกลางคณะวิทยาศาสตร์    & 0 & 0 & 0 & 0 & 0 & 0 & 0 & 0 & 0 & 0 & 0 & 0 \\
                โรงอาหารคณะรัฐศาสตร์ (ตรงข้าม)      & 0 & 0 & 0 & 0 & 0 & 0 & 0 & 0 & 0 & 0 & 0 & 0 \\
                ไปรษณีย์                          & 0 & 0 & 0 & 0 & 0 & 0 & 0 & 0 & 0 & 0 & 0 & 0 \\
                ลานจอดรถ อ่างแก้ว                  & 0 & 0 & 0 & 0 & 0 & 0 & 0 & 0 & 0 & 0 & 0 & 0 \\
                โรงอาหารคณะมนุษยศาสตร์            & 0 & 0 & 0 & 0 & 0 & 0 & 0 & 0 & 0 & 0 & 0 & 0 \\
                อาคาร HB7 คณะมนุษยศาสตร์          & 0 & 0 & 0 & 0 & 0 & 0 & 0 & 0 & 0 & 0 & 0 & 0 \\
                สำนักหอสมุด                       & 0 & 0 & 0 & 0 & 0 & 0 & 0 & 0 & 0 & 0 & 0 & 0 \\
                สถานีกลางรถไฟฟ้า ขสมช.            & 0 & 0 & 0 & 0 & 0 & 0 & 0 & 0 & 0 & 0 & 0 & 0 \\
            \hline
        \end{tabular}
    \end{center}}

    \textbf{เวลา 10:30 น.}
    {\tiny\begin{center}
        \begin{tabular}{ | c | c | c | c | c | c | c | c | c | c | c | c | c |  }
            \hline
                \multirow{2}{*}{สถานี} & \multicolumn{3}{|c|}{ความหนาแน่นจริง (คน)} & \multicolumn{3}{|c|}{ความหนาแน่นบนระบบ (คน)} & \multicolumn{3}{|c|}{ผลต่าง (คน)} & \multicolumn{3}{|c|}{ร้อยละที่คลาดเคลื่อน (\%)} \\
            \hline
                & เข้า & ออก & ปัจจุบัน & เข้า & ออก & ปัจจุบัน & เข้า & ออก & ปัจจุบัน & เข้า & ออก & ปัจจุบัน \\
            \hline
                สถานีกลางรถไฟฟ้า ขสมช.            & 1 & 0 & 1 & 1 & 0 & 1 & 0 & 0 & 0 & 0 & 0 & 0 \\
                อาคารปฏิบัติการกลางคณะวิทยาศาสตร์    & 1 & 0 & 1 & 1 & 0 & 1 & 0 & 0 & 0 & 0 & 0 & 0 \\
                โรงอาหารคณะรัฐศาสตร์ (ตรงข้าม)      & 1 & 0 & 1 & 1 & 0 & 1 & 0 & 0 & 0 & 0 & 0 & 0 \\
                ไปรษณีย์                          & 1 & 0 & 1 & 1 & 0 & 1 & 0 & 0 & 0 & 0 & 0 & 0 \\
                ลานจอดรถ อ่างแก้ว                  & 1 & 0 & 1 & 1 & 0 & 1 & 0 & 0 & 0 & 0 & 0 & 0 \\
                โรงอาหารคณะมนุษยศาสตร์            & 1 & 0 & 1 & 1 & 0 & 1 & 0 & 0 & 0 & 0 & 0 & 0 \\
                อาคาร HB7 คณะมนุษยศาสตร์          & 2 & 0 & 2 & 2 & 0 & 2 & 0 & 0 & 0 & 0 & 0 & 0 \\
                สำนักหอสมุด                       & 2 & 1 & 1 & 2 & 1 & 1 & 0 & 0 & 0 & 0 & 0 & 0 \\
                สถานีกลางรถไฟฟ้า ขสมช.            & 2 & 2 & 0 & 2 & 1 & 1 & 0 & 0 & 0 & 0 & 6.25 & 6.25 \\
            \hline
        \end{tabular}
    \end{center}}

    เฉลี่ยแล้วกิจกรรมการเข้า-ออกรวมคลาดเคลื่อนเฉลี่ยร้อยละ \textbf{0.11} และ จำนวนผู้โดยสารบนรถคลาดเคลื่อนเฉลี่ยร้อยละ \textbf{0.23}

ผู้จัดทำโครงงานตั้งเป้าหมายว่าระบบใหม่จะต้องมีความคลาดเคลื่อนเฉลี่ยและสูงสดน้อยกว่าร้อยละ 20 (ประมาณ ± 3 คน สําหรับรถ 16 ที่นั่ง) พบว่าค่าคลาดเคลื่อนเฉลี่ยต่ำกว่าเป้าหมาย แต่ความคลาดเคลือนสูงสุดสูงกว่าเป้าหมาย อย่างไรก็ตาม การวัดผลยังตัดสินประสิทธิภาพได้ไม่เต็มที่ เนื่องจากระหว่างการทดสอบมีการปรับปรุงระบบเนื่องจากข้อผิดพลาดที่พบ และยังไม่ได้ทดสอบกับเหตุการณ์ที่ผู้โดยสารคับคั่งมากกว่านี้

\subsection{ความเสถียรของข้อมูล}

ตรวจสอบการส่งข้อมูลผ่านเว็บแอพพลิเคชัน

ผู้จัดทำโครงงานตั้งเป้าหมายว่าระบบใหม่จะต้องคลอบคลุมตลอดช่วงร้อยละ 85 ของการให้บริการ (7.65 ชั่วโมง จาก 9 ชั่วโมง) แต่จากการตรวจสอบพบว่ามีการส่งข้อมูลเพียง 2 ชั่วโมง จาก 9 ชั่วโมง หรือร้อยละ 22.22 ซึ่งยังไม่ครอบคลุมตลอดช่วงเวลาที่กำหนด จากการสำรวจ พบว่าเกิดจากปัญหาที่อุปกรณ์ใช้กระแสไฟฟ้ามากเกินไป รวมไปถึงแบตเตอรีที่ใช้งานนั้นเสื่อมสถาพ ทำให้ไม่ครอบคลุมตามที่คำนวณไว้
