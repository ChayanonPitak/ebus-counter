\chapter{\ifenglish Background Knowledge and Theory\else ทฤษฎีที่เกี่ยวข้อง\fi}

การทำโครงงาน เริ่มต้นด้วยการศึกษาค้นคว้า ทฤษฎีที่เกี่ยวข้อง หรือ งานวิจัย/โครงงาน ที่เคยมีผู้นำเสนอไว้แล้ว ซึ่งเนื้อหาในบทนี้ก็จะเกี่ยวกับการอธิบายถึงสิ่งที่เกี่ยวข้องกับโครงงาน เพื่อให้ผู้อ่านเข้าใจเนื้อหาในบทถัดๆ ไปได้ง่ายขึ้น

\section{รถไฟฟ้าของขนส่งมวลชนมหาวิทยาลัยเชียงใหม่}
รถรับส่งไฟฟ้าที่มี16ที่นั่ง (ไม่รวมคนขับ) ที่ให้บริการนักศึกษาและบุคลากรของมหาวิทยาลัยเชียงใหม่ โดยจะให้บริการตามสายที่กำหนดไว้ โดยให้บริการในช่วง 06:00 น. ถึง 21:00 น. หรือน้อยกว่า ตามที่ผู้ให้บริการกำหนด

\section{ไมโครคอลโทรลเลอร์ ESP 32}
ESP 32 เป็นไมโครคอนโทรลเลอร์ที่ใช้พลังงานต่ำและสามารถความคุมรวมไปถึงรับข้อมูลจากอุปกรณ์และระบบต่างๆได้หลากหลาย

\section{โมเดม SimCOM T-A7608 และบอร์ดสำหรับพัฒนา Lilygo T-A7608}
SimCOM T-A7608 เป็นโมเดมที่สามารถเชื่อมต่อกับเครือข่าย 4G และ Wi-Fi , เชื่อมต่อกับ GPS รวมไปถึงเชื่อมต่อกับโปรโตคอลอื่น อาทิ HTTP, MQTT, Websocket, ฯลฯ เพื่อรับ-ส่งข้อมูลภายในเครือข่ายได้
บอร์ดสำหรับพัฒนา Lilygo T-A7608 เป็นบอร์ดสำหรับพัฒนาที่รวมความหลากหลายในการทำงานของ ไมโครคอลโทรลเลอร์ ESP 32 และความสามารถในการเชื่อมต่อเครือข่ายของโมเดม SimCOM T-A7608 เข้าด้วยกัน และสามารถใช้พัฒนาเสมือนเป็นไมโครคอนโทรลเลอร์อันหนึ่งได้


\section{เซนเซอร์วัดระยะทางแบบอินฟาเรด Sharp GP2Y0A02YK0F}
Sharp GP2Y0A02YK0F เป็นเซนเซอร์ที่ใช้สำหรับวัดระยะห่างจากตัวเซนเซอร์ ซึ่งตัวเซนเซอร์ประกอบไปด้วยหลอดไอโอดแสงอินฟราเรด สำหรับยิงลำแสงเพื่อให้กระทบกับวัตถุและสะท้อนกลับมา, Position Sensitive Detector หรือ PSD สำหรับรับแสงอินฟราเรดบนระนาบรับแสง และส่งข้อมูลทางไฟฟ้าตามตำแหน่งที่แสงอินฟราเรดตกกระทบบนอุปกรณ์ตรวจวัด และ Signal Processing Unit สำหรับส่งข้อมูลทางไฟฟ้าออกจากดซนเซอร์ตามข้อมูลของ PSD โดยเซนเซอร์วัดระยะทางแบบอินฟาเรด Sharp GP2Y0A02YK0F นั้นสามารถวัดระยะทางได้ดีที่ 20 ถึง 150 เซนติเมตร

\section{4G}
4G เป็นระบบเครือข่ายอินเตอน์เน็ตไร้สายระยะไกลที่พัฒนาต่อมาจาก 3G และ 2G

\section{Global Positioning System}
Global Positioning System หรือ GPS เป็นระบบดาวเทียมนำร่องโลก เพื่อระบุข้อมูลของตำแหน่งและเวลาโดยอาศัยการคำนวณจากความถี่สัญญาณนาฬิกาที่ส่งมาจากตำแหน่งของดาวเทียมต่างๆ ที่โคจรอยู่รอบโลกทำให้สามารถระบุตำแหน่ง ณ จุดที่สามารถรับสัญญาณได้ทั่วโลกและในทุกสภาพอากาศ รวมถึงสามารถคำนวณความเร็วและทิศทางเพื่อนำมาใช้ร่วมกับแผนที่ในการนำทางได้

% \subsection{Subsection heading goes here}

% Subsection 1 text

% \subsubsection{Subsubsection 1 heading goes here}
% Subsubsection 1 text

% \subsubsection{Subsubsection 2 heading goes here}
% Subsubsection 2 text

\section{MQTT}
MQTT เป็นโปรโตคอลการส่งข้อความที่อิงตามมาตรฐาน หรือชุดของกฎที่ใช้สําหรับการสื่อสารระหว่าง เครื่องต่อเครื่อง เซ็นเซอร์อัจฉริยะ อุปกรณ์สวมใส่ และอุปกรณ์Internet of Things (IoT) อื่นๆ มักจะ ต้องส่งและรับข้อมูลผ่านเครือข่ายที่มีข้อจํากัดด้านทรัพยากร ซึ่งมีแบนด์วิดท์จํากัด อุปกรณ์IoT เหล่านี้ใช้ MQTT ในการรับส่งข้อมูล เนื่องจากมันใช้งานง่ายและสามารถสื่อสารข้อมูล IoT ได้อย่างมีประสิทธิภาพ MQTT รองรับการส่งข้อความจากอุปกรณ์ไปยังคลาวด์และจากคลาวด์ไปยังอุปกรณ์

\section{REST API}
Representational State Transfer (REST) เป็นสถาปัตยกรรมซอฟต์แวร์ที่กําหนดเงื่อนไขว่า API ควร ทํางานอย่างไร โดยแต่แรกเริ่มนั้น มีการสร้าง REST ขึ้นเพื่อเป็นแนวทางในการจัดการการสื่อสารบนเครือ ข่ายที่ซับซ้อน เช่น อินเทอร์เน็ต คุณสามารถใช้สถาปัตยกรรม REST เพื่อรองรับการสื่อสารที่มีประสิทธิภาพ สูงและเชื่อถือได้ในทุกระดับ คุณยังสามารถใช้และปรับเปลี่ยนสถาปัตยกรรมได้อย่างง่ายดาย โดยนําความสามารถในการมองเห็นและการเคลื่อนย้ายข้ามแพลตฟอร์มมาสู่ทุกระบบ API

% \section{About using figures in your report}

% % define a command that produces some filler text, the lorem ipsum.
% \newcommand{\loremipsum}{
%   \textit{Lorem ipsum dolor sit amet, consectetur adipisicing elit, sed do
%     eiusmod tempor incididunt ut labore et dolore magna aliqua. Ut enim ad
%     minim veniam, quis nostrud exercitation ullamco laboris nisi ut
%     aliquip ex ea commodo consequat. Duis aute irure dolor in
%     reprehenderit in voluptate velit esse cillum dolore eu fugiat nulla
%     pariatur. Excepteur sint occaecat cupidatat non proident, sunt in
%     culpa qui officia deserunt mollit anim id est laborum.}\par}

% \begin{figure}
%   \centering

%   \fbox{
%     \parbox{.6\textwidth}{\loremipsum}
%   }

%   % To include an image in the figure, say myimage.pdf, you could use
%   % the following code. Look up the documentation for the package
%   % graphicx for more information.
%   % \includegraphics[width=\textwidth]{myimage}

%   \caption[Sample figure]{This figure is a sample containing \gls{lorem ipsum},
%     showing you how you can include figures and glossary in your report.
%     You can specify a shorter caption that will appear in the List of Figures.}
%   \label{fig:sample-figure}
% \end{figure}

% Using \verb.\label. and \verb.\ref. commands allows us to refer to
% figures easily. If we can refer to Figures
% \ref{fig:walrus} and \ref{fig:sample-figure} by name in the {\LaTeX}
% source code, then we will not need to update the code that refers to it
% even if the placement or ordering of the figures changes.

% \loremipsum\loremipsum

% % This code demonstrates how to get a landscape table or figure. It
% % uses the package lscape to turn everything but the page number into
% % landscape orientation. Everything should be included within an
% % \afterpage{ .... } to avoid causing a page break too early.
% \afterpage{
%   \begin{landscape}
%     \begin{table}
%       \caption{Sample landscape table}
%       \label{tab:sample-table}

%       \centering

%       \begin{tabular}{c||c|c}
%         Year & A  & B  \\
%         \hline\hline
%         1989 & 12 & 23 \\
%         1990 & 4  & 9  \\
%         1991 & 3  & 6  \\
%       \end{tabular}
%     \end{table}
%   \end{landscape}
% }

% \loremipsum\loremipsum\loremipsum

% \section{Overfull hbox}

% When the \verb.semifinal. option is passed to the \verb.cpecmu. document class,
% any line that is longer than the line width, i.e., an overfull hbox, will be
% highlighted with a black solid rule:
% \begin{center}
%   \begin{minipage}{2em}
%     juxtaposition
%   \end{minipage}
% \end{center}

% \section{\ifenglish%
%     \ifcpe CPE \else ISNE \fi knowledge used, applied, or integrated in this project
%   \else%
%     ความรู้ตามหลักสูตรซึ่งถูกนำมาใช้หรือบูรณาการในโครงงาน
%   \fi
%  }

% อธิบายถึงความรู้ และแนวทางการนำความรู้ต่างๆ ที่ได้เรียนตามหลักสูตร ซึ่งถูกนำมาใช้ในโครงงาน

% \section{\ifenglish%
%     Extracurricular knowledge used, applied, or integrated in this project
%   \else%
%     ความรู้นอกหลักสูตรซึ่งถูกนำมาใช้หรือบูรณาการในโครงงาน
%   \fi
%  }

% อธิบายถึงความรู้ต่างๆ ที่เรียนรู้ด้วยตนเอง และแนวทางการนำความรู้เหล่านั้นมาใช้ในโครงงาน
