\maketitle
\makesignature

\ifproject
\begin{abstractTH}
    ในระบบขนส่งมวลชนต่างๆ การมีข้อมูลต่างๆ ที่เกี่ยวข้องกับการใช้บริการขนส่งมวลชน จะช่วยให้ผู้ให้บริการขนส่งมวลชนนำข้อมูลที่มีเพื่อจัดการการให้บริการอย่างเหมาะสมและพึงพอใจกับผู้ใช้บริการได้ โดยหนึ่งในข้อมูลที่จำเป็นต่อการให้บริการคือ ความหนาแน่นของผู้โดยสารระหว่างการให้บริการ เราจึงจัดทำ ระบบตรวจสอบความหนาแน่นบนรถไฟฟ้าของขนส่งมวลชนมหาวิทยาลัยเชียงใหม่ ซึ่งเป็นซอฟต์แวร์และชุดอุปกรณ์ที่ตรวจสอบผู้ใช้บริการรถโดยสาร ณ เวลาหนึ่ง, สื่อสารกับระบบที่ผู้พัฒนาโครงการจัดทำขึ้น เพื่อวิเคราห๋และแสดงผลข้อมูลที่ได้จากการวัด
\end{abstractTH}

\begin{abstract}
    In the public transportation system, having various data related to the use of public transportation services will help the service provider manage the service appropriately and satisfy the users. One of the necessary data for service provision is the occupancy during the service. We, therefore, developed a system to check the bus occupancy for the Chiang Mai University transit electric bus. The system is software and a set of devices that check the passenger on the bus at a specific time and communicate with the system developed by the project developer, to analyze and display the data obtained from the measurement.
\end{abstract}

\iffalse
\begin{dedication}
This document is dedicated to all Chiang Mai University students.

Dedication page is optional.
\end{dedication}
\fi % \iffalse

\begin{acknowledgments}
    โครงงานเรื่อง ระบบตรวจสอบความหนาแน่นบนรถไฟฟ้าของขนส่งมวลชนมหาวิทยาลัยเชียงใหม่ เพื่อการสำเร็จการศึกษาของนักศึกษาระดับปริญญาตรี สามารถสำเร็จลุล่วงไปได้ด้วยดี เนื่องจากได้รับความอนุเคราห์จาก ผู้ช่วยศาสตราจารย์ ดร. ภาสกร แช่มประเสริฐ อาจารย์ที่ปรึกษา, ผู้ช่วยศาสตราจารย์ ดร. กําพล วรดิษฐ์, และ รองศาสตราจารย์ ดร. สันติ พิทักษ์กิจนุกูร คณะกรรมการที่ปรึกษาโครงงาน ที่ได้กรุณาให้คำปรึกษา คำแนะนำ ความรู้ และการสนับสนุนอื่นๆ ตลอดระยะเวลาการศึกษา จนโครงงานนี้สามารถสำเร็จลุล่วงไปได้ด้วยดี 

    ขอขอบคุณ ศูนย์บริหารจัดการเมืองอัจฉริยะมหาวิทยาลัยเชียงใหม่ ที่ให้ความอนุเคราะห์ในการให้ข้อมูลการเดินรถของรถไฟฟ้าของขนส่งมวลชนมหาวิทยาลัยเชียงใหม่ รวมถึงอนุเคราห์ให้ติดตั้งชุดอุปกรณ์บนรถไฟฟ้าของขนส่งมวลชนมหาวิทยาลัยเชียงใหม่ชั่วคราว

    ขอขอบคุณกลุ่มงานวิจัย OASYS ที่ให้ความอนุเคราห์อุปกรณ์ต่างๆที่ใช้ในการพัฒนาโครงงาน

    สุดท้ายนี้ผู้พัฒนาโครงงานหวังว่าโครงงานนี้จะเป็นประโยชน์สำหรับสำหรับหน่วยงานที่เกี่ยวข้องและ ผู้ที่สนใจศึกษาต่อไป
\acksign{2024}{3}{25}
\end{acknowledgments}%
\fi % \ifproject

\contentspage

\ifproject
\figurelistpage

\tablelistpage
\fi % \ifproject

% \abbrlist % this page is optional

% \symlist % this page is optional

% \preface % this section is optional
