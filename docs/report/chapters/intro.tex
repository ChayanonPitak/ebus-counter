\chapter{\ifenglish Introduction\else บทนำ\fi}

\section{\ifenglish Project rationale\else ที่มาของโครงงาน\fi}

ในการให้บริการขนส่งมวลชนในปัจจุบัน ผู้ให้บริการจําเป็นต้องมีข้อมูลเกี่ยวกับบริการที่ถูกต้องและแม่นยํา
ซึ่งหนึ่งในข้อมูลนั้นคือข้อมูลความหนาแน่นของผู้โดยสาร ณ เวลาหนึ่ง ซึ่งบริการรถไฟฟ้าของขนส่งมวลชน
มหาวิทยาลัยเชียงใหม่ก็มีระบบวัดข้อมูลความหนาแน่นเช่นกัน โดยทํางานภายใต้พื้นฐานของการประมวลผล
ภาพจากกล้อง แต่ว่าข้อมูลที่ได้นั้นไม่แม่นยํามากพอ เนื่องจากปัญหาสภาพแวดล้อม ซึ่งไม่สามารถควบคุมได้ ผู้
จัดทําโครงงานจึงต้องการจัดทําระบบใหม่เพื่อเพิ่มความแม่นยําของการเก็บข้อมูลและแสดงผลความหนาแน่นของผู้โดยสารผ่าน
โครงงานนี้

\section{\ifenglish Objectives\else วัตถุประสงค์ของโครงงาน\fi}
\begin{enumerate}
    \item พัฒนาระบบวัดความหนาแน่นของระบบขนส่งมวลชนของรถไฟฟ้าของขนส่งมวลชนมหาวิทยาลัย
    เชียงใหม่ให้แม่นยํามากกว่าระบบที่มีอยู่เดิมและใช้งานได้จริงในสภาพแวดล้อมจริง
    \item พัฒนาระบบวัดความหนาแน่นของรถโดยสารที่มีความแม่นยำสูง และต้นทุนต่ำ
    \item พัฒนาเว็บแอพพลิเคชันที่สามารถแสดงผลข้อมูลความหนาแน่นของผู้โดยสารจากระบบที่พัฒนาข้างต้น
\end{enumerate}

\section{\ifenglish Project scope\else ขอบเขตของโครงงาน\fi}

\subsection{\ifenglish Hardware scope\else ขอบเขตด้านฮาร์ดแวร์\fi}
\begin{enumerate}
    \item อุปกรณ์ที่ใช้สำหรับวัดความหนาแน่นของผู้โดยสารนั้นจะพัฒนาและมดสอบการติดตั้งบนรถไฟฟ้าของขนส่งมวลชนมหาวิทยาลัยเชียงใหม่เท่านั้น ซึ่งเป็นรถไฟฟ้าขนาด 16 ที่นั่ง (ไม่รวมที่นั่งคนขับ) และมีทางเข้าออกของผู้โดยสาร 2 ประตู
    \item อุปกรณ์สําหรับวัดความหนาแน่นจะวัดได้อย่างถูกเฉพาะการโดยสารของมนุษย์ ไม่รวมการโดยสารของ
    สัตว์เลี้ยง หรือวัตถุใดๆที่ไม่ใช่มนุษย์
    \item อุปกรณ์สําหรับวัดความหนาแน่นจะวัดได้อย่างถูกต้องเฉพาะการเข้า-ออกรถโดยสารผ่านประตูสําหรับ
    ผู้โดยสารเท่านั้น โดยต้องเข้า-ออกได้มากที่สุดประตูละ 1 คนต่อครั้ง
    \item ระบบวัดความหนาแน่นที่พัฒนาขั้นจะสามารถทํางานได้ถูกต้องในช่วงเวลาที่รถไฟฟ้าที่ถูกติดตั้งอยู่ในระหว่างการให้บริการ (07:00 น. - 22:00 น.)
    \item พื้นที่อุปกรณ์ที่พัฒนาขึ้นทํางานอยู่จะต้องไม่ถูกรบกวนสัญญาณเซลลูราร์มากเกินกว่า
    ระดับสิ่งแวดล้อมทั่วไป
    \item อุปกรณ์ที่พัฒนาขึ้นจะไม่รบกวนการทำงานใดๆของรถโดยสารที่ถูกติดตั้งอยู่
\end{enumerate}

\subsection{\ifenglish Software scope\else ขอบเขตด้านซอฟต์แวร์\fi}
\begin{enumerate}
    \item เว็บแอพพลิเคชันที่พัฒนาขึ้นจะสามารถแสดงผลข้อมูลความหนาแน่นของผู้โดยสารจากระบบที่พัฒนาข้างต้นเท่านั้น
    \item เว็บแอพพลิเคชันที่พัฒนาขึ้นจะแสดงผลได้อย่างถูกต้องบนเดสก์ท็อปเท่านั้น ไม่รองรับการแสดงผลบนอุปกรณ์พกพาที่ไม่ใช้แลปท็อป
    \item เว็บแอพพลิเคชันที่พัฒนาขึ้นจะสามารถแสดงผลได้อย่างถูกต้องบนเบราว์เซอร์ที่รุ่นสูงกว่า หรือเทียบเท่ากับ Google Chrome รุ่น 45, Firefox รุ่น 34, Safari รุ่น 9, และ Microsoft Edge รุ่น 12 หรือเว็บเบาเซอร์อื่นๆที่การรองรับเทียบเท่ากับเบราว์เซอร์ที่กล่าวมา
\end{enumerate}

\section{\ifenglish Expected outcomes\else ประโยชน์ที่ได้รับ\fi}
\textbf{สำหรับผู้ให้บริการรถโดยสาร}
\begin{enumerate}
    \item สามารถทราบความหนาแน่นของผู้โดยสารในแต่ละคัน, สถานี, สาย และช่วงเวลาได้อย่างแม่นยำโดยไม่ระบุตัวตนของผู้โดยสาร เพื่อนำไปปรับปรุงและพัฒนาระบบโดยสารให้ดียิ่งขึ้น
\end{enumerate}
\textbf{สำหรับผู้รอรถโดยสาร}
\begin{enumerate}
    \item สามารถทราบความหนาแน่นของผู้โดยสารในแต่ละคัน, สถานี และสาย ในปัจจุบันได้ เพื่อนำไปวางแผนการเดินทางตามข้อมูลความหนาแน่นที่ได้รับ
\end{enumerate}

\section{\ifenglish Technology and tools\else เทคโนโลยีและเครื่องมือที่ใช้\fi}

\subsection{\ifenglish Hardware technology\else เทคโนโลยีด้านฮาร์ดแวร์\fi}
\begin{enumerate}
    \item บอร์ดสำหรับพัฒนาระบบสมองกลฝังตัว LilyGo T-A7608E-H แบบไม่มี GNSS เป็นหน่วยประมวลผลหลักของอุปกรณ์วัด ซึ่งเป็นการรวมกันของไมโครคอนโทรลเลอร์ ESP32 ซึ่งเป็นไมโครคอนโทรลเลอร์ที่นิยมใช้ในโครงงานต่างๆ และโมเด็ม SIMCOM A7608E-H สำหรับเชื่อมต่ออินเตอร์เน็ตผ่านเครือข่ายเซลลูลาร์ 
    \item อุปกรณ์วัดระยะห่างโดยใช้อินฟราเรด Sharp GP2Y0A02YK0F สำหรับวัดค่าความหนาแน่นของผู้โดยสาร
    \item อุปกรณ์รับ-ส่งข้อมูล GNSS u-blox NEO-7M สำหรับระบุตำแหน่งของรถโดยสาร
    \item คอมพิวเตอร์ส่วนตัวสำหรับพัฒนาโปรแกรม
\end{enumerate}

\subsection{\ifenglish Software technology\else เทคโนโลยีด้านซอฟต์แวร์\fi}
\begin{enumerate}
    \item PlatformIO สำหรับการพัฒนาโปรแกรมบนระบบสมองกลฝังตัว
    \item Mosquitto สำหรับการรับส่งข้อมูลผ่านโปรโตคอล MQTT
    \item Node.js สำหรับการพัฒนาการเชื่อมต่อระหว่างโบรกเกอร์ MQTT, ฐานข้อมูล และเว็บแอพพลิเคชัน
    \item React.js สำหรับการพัฒนาเว็บแอพพลิเคชัน
    \item คลาวด์ DigitalOcean สำหรับการให้บริการเว็บแอพพลิเคชัน ฐานข้อมูล โบรกเกอร์ MQTT และระบบเชื่อมต่อ
\end{enumerate}

\section{\ifenglish Project plan\else แผนการดำเนินงาน\fi}

\textbf{ภาคการศึกษาที่ 2/2565}
\begin{plan}{1}{2022}{3}{2022}
    \planitem{1}{2022}{2}{2022}{ศึกษาค้นคว้าเกี๋ยวกับหัวข้อและข้อมูลต่างๆที่เกี่ยวข้อง}
    \planitem{2}{2022}{3}{2022}{ออกแบบอุปกรณ์สำหรับวัดความหนาแน่น}
    \planitem{2}{2022}{3}{2022}{ออกแบบระบบเชื่อมต่อระหว่างระบบ}
\end{plan}

\textbf{ภาคการศึกษาที่ 2/2566}
\begin{plan}{10}{2023}{3}{2024}
    \planitem{10}{2023}{3}{2024}{ออกแบบอุปกรณ์สำหรับวัดความหนาแน่น}
    \planitem{11}{2023}{3}{2024}{จัดทำอุปกรณ์สำหรับทดสอบ}
    \planitem{12}{2023}{3}{2024}{ทดสอบวัดความหนาแน่นและการส่งข้อมูลบนสภาพแวดล้อมจำลอง}
    \planitem{3}{2024}{3}{2024}{ทดสอบและประเมินผลอุปกรณ์วัดความหนาแน่นและการส่งข้อมูลบนสภาพแวดล้อมจำลองจริง}
    \planitem{10}{2023}{12}{2023}{ออกแบบระบบเชื่อมต่อระหว่างระบบ}
    \planitem{10}{2023}{12}{2023}{ออกแบบประสบการณ์ผู้ใช้และส่วนติดต่อผู้ใช้}
    \planitem{12}{2023}{3}{2024}{พัฒนาเว็บแอพพลิเคชัน}
    \planitem{1}{2024}{3}{2024}{พัฒนาระบบต่างๆบนคลาวด์}
    \planitem{3}{2024}{3}{2024}{ประเมินผลเว็บแอพพลิเคชัน}
    \planitem{3}{2024}{3}{2024}{สรุปผลและจัดทำรายงาน}
\end{plan}

\section{\ifenglish Roles and responsibilities\else บทบาทและความรับผิดชอบ\fi}

นายกิจพิสันต์ ทันงาน รหัสนักศึกษา 630610716 รับผิดชอบในการพัฒนาระบบเชื่อมต่อระหว่างระบบ ฐานข้อมูล และเว็บแอพพลิเคชัน
นายชญานนท์ พิทักษ์ รหัสนักศึกษา 630610724 รับผิดชอบในการพัฒนาอุปกรณ์สำหรับวัดความหนาแน่น
\section{\ifenglish%
Impacts of this project on society, health, safety, legal, and cultural issues
\else%
ผลกระทบด้านสังคม สุขภาพ ความปลอดภัย กฎหมาย และวัฒนธรรม
\fi}

หากโครงการนี้ประสบผลสําเร็จและมีการนําไปต่อยอดแล้ว จะมีผลกระทบต่อการให้บริการรถ
โดยสารต่างๆที่ต้องการข้อมูลความหนาแน่นของผู้โดยสาร เพื่อนำไปพัฒนาการให้บริการที่ตรงต่อความต้องการของผู้ใช้บริการระบบโดยสารมากขึ้นได้ รวมไปถึงให้ผู้ใช้บริการสามารถวางแผยการเดินทางได้อย่างมีประสิทธิภาพมากขึ้น

