\chapter{The first appendix}

% Text for the first appendix goes here.

\section{Appendix section}

% Text for a section in the first appendix goes here.

% test ทดสอบฟอนต์ serif ภาษาไทย

% \textsf{test ทดสอบฟอนต์ sans serif ภาษาไทย}

% \verb+test ทดสอบฟอนต์ teletype ภาษาไทย+

% \texttt{test ทดสอบฟอนต์ teletype ภาษาไทย}

% \textbf{ตัวหนา serif ภาษาไทย \textsf{sans serif ภาษาไทย} \texttt{teletype ภาษาไทย}}

% \textit{ตัวเอียง serif ภาษาไทย \textsf{sans serif ภาษาไทย} \texttt{teletype ภาษาไทย}}

% \textbf{\textit{ตัวหนาเอียง serif ภาษาไทย \textsf{sans serif ภาษาไทย} \texttt{teletype ภาษาไทย}}}

% \url{https://www.example.com/test_ทดสอบ_url}

\chapter{\ifenglish Manual\else คู่มือการใช้งานระบบ\fi}

\section{การติดตั้งเซิฟเวอร์}
    ทำการ git clone ที่ \url{https://github.com/DifficultIV/261492-Backend} จากนั้นให้ทำการพิมพ์คำสั่ง npm install ลงใน terminal เพื่อทำการโหลดข้อมูลที่ต้องใช้(ใช้คำสั่งนี้เพียงครั้งแรกเท่านั้น) และพิมพ์คำสั่ง node --env-file=.env server.js เพื่อทำการเริ่มการทำงานของเซิฟเวอร์

\section{การติดตั้งเว็บไซต์}    
    ทำการ git clone ที่ \url{https://github.com/DifficultIV/261492-Occupancy-monitoring} จากนั้นให้ทำการพิมพ์คำสั่ง cd react-app ลงใน terminal เพื่อเข้าไปยังโฟลเดอร์ react-app จากนั้นให้ทำการพิมพ์คำสั่ง npm install เพื่อทำการโหลดข้อมูลที่ต้องใช้(ใช้คำสั่งนี้เพียงครั้งแรกเท่านั้น) และพิมพ์คำสั่ง npm start เพื่อทำการเริ่มการทำงานของเว็บไซต์